\section{How to Use this book}\label{sec:how-to-use}
My goal is that at the end of this book, you will have built a web app.
You will find that the fastest way to do this is to do-and-learn.
This is not a book to read passively like a novel.
It requires you to have conversation with it, to try the examples,
to ask questions, try a few ideas; this is the fastest way to learn coding.

Let me break the steps down.
%
\subsection{Start with 15 minutes}\label{subsec:start-with-15-minutes}
Find the smallest time you can dedicate to building the project in this book every day.
Yes, I've said this a few times.\ It needs every day dedication.
I recommend starting with 15 minutes.

\subsection{How to "Try It"}\label{subsec:the-trying-it}
In your 15-minute block you can either read a chapter, a section or a subsection.
When you get to the section that has a ``Try it\ldots''; turn off your phone, open up your laptop and try it.

This is the most important part of your progress.
Taking your time to actually try the exercises.

And when you're stuck or need extra help?

\subsection{Get help from me and the community}\label{subsec:get-help-from-me}

You can get help from me and my community (\emph{Your Next Best Move}) on discord.
Post the section you need help with and I or other learners will be happy to help you along.

Now let's go to our first "try it."


\section{Try it: Join my community}\label{sec:try-it: lajoin-my-community}

\begin{enumerate}
    \item Go to \url{https://to.dto.sh/ynbm-discord}
    \item Enter your \textbf{Display Name} and Continue
    \item You can optionally create a discord account
    \item Go to the \#coding channel and say hello (of course you can @deletosh)
\end{enumerate}
